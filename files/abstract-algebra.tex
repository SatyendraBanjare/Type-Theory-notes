\chapter{Abstract Algebra }

Abstract algebra also referred to as modern algebra is the study of algebraic structures. An algebric structure serves as an explinatory basis of functional operations on an underlying set. A set here also is an abstract idea of a collection of things that share certain common features and should not only be confined with sets dealt in classical set theory. Over the time on the basis of types of operations and \textit{logical freedom} to do so on various sets have led to acceptance of various algebric structures defined below. The study of abstract algebra is used primarily in areas of topology of \textbf{n} dimensions. For anyone this is insane for physical boundations arise when we try to visualize things out. Thus we need to classify abstractions in one of the algebric structures and then deal with it. 

\section{Group Theory}

This is a part of abstract algebra where we deal with \textbf{Group} mathematical structures. In mathematics, a \textbf{Group} is an abstract algebraic structure that consists of a set of elements and various operations which when performed on any \textit{two} elements of the set results in a \textit{third} element from same set. It satisfies four conditions called the "group axioms" or "group properties", namely closure or closed operations i.e. that maps from set to itself , associativity, identity and invertibility. One of the most common examples of a group structure is the set of integers with the addition operation. This algebric structure is fundamental basis of other more complex algebric structures. It is studied in followings ways.\\

\subsection{Cyclic Groups}

\lipsum[1]

\subsection{Permutation groups}
\subsection{Group Actions}

\lipsum[1]
\lipsum[1]
\section{Field Theory}

\section{Rings Theory}
\subsection{Polynomial Rings}

\section{Iso-morphisms}

\section{Homo-morphisms}

\section{Matrix Groups}

\section{Category Theory}

Algebraic structures, with their associated homomorphisms, form mathematical categories. Category theory is a formalism that allows a unified way for expressing properties and constructions that are similar for various structures. \\

The language of category theory is used to express and study relationships between different classes of algebraic and non-algebraic objects. This is because it is sometimes possible to find strong connections between some classes of objects, sometimes of different kinds. For example, Galois theory establishes a connection between certain fields and groups: two algebraic structures of different kinds.\\