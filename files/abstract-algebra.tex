\chapter{Abstract Algebra }

Abstract algebra also referred to as modern algebra is the study of algebraic structures. An algebric structure serves as an explinatory basis of functional operations on an underlying set. A set here also is an abstract idea of a collection of things that share certain common features and should not only be confined with sets dealt in classical set theory. Over the time on the basis of types of operations and \textit{logical freedom} to do so on various sets have led to acceptance of various algebric structures defined below. The study of abstract algebra is used primarily in areas of topology of \textbf{n} dimensions. For anyone this is insane for physical boundations arise when we try to visualize things out. Thus we need to classify abstractions in one of the algebric structures and then deal with it. 

\section{Group Theory}

This is a part of abstract algebra where we deal with \textbf{Group} mathematical structures. In mathematics, a \textbf{Group} is an abstract algebraic structure that consists of a set of elements and various operations which when performed on any \textit{two} elements of the set results in a \textit{third} element from same set. It satisfies four conditions called the "group axioms" or "group properties", namely closure or closed operations i.e. that maps from set to itself , associativity, identity and invertibility. One of the most common examples of a group structure is the set of integers with the addition operation. This algebric structure is fundamental basis of other more complex algebric structures. It is studied in followings ways.\\

A group \textit{G} with the property with a given operation "o" such that \[ \textit{ a o b = b o a }  | \hfill \forall a,b \in \{ G \} \] is called abelian or commutative. Groups not satisfying this property are said to be nonabelian or non-commutative.

\subsection{Cyclic Groups}

A \textbf{Cyclic Group} is a group that can be generated by a single element often called as group generator). Cyclic groups are always Abelian. A cyclic group of finite group order n is denoted $G_n$ . The genralized generation rule can be specified as : \\

\[ X^n  = I ( X \in G ) \]

In the simple sense, it means identity element can be generated by any single element by repeated application of group operations. And since the identity element can be realized this way, all the elements of group can be realized too.\\

\subsection{Permutation groups}

A \textbf{Permutation Group} is a group G whose elements are permutations of a given set M and whose group operation is the composition of permutations in G (which are thought of as bijective functions from the set M to itself). The group of all permutations of a set M is the symmetric group of M, often written as Sym(M).[1] The term permutation group thus means a subgroup of the symmetric group. If M = {1,2,...,n} then, Sym(M), the symmetric group on n letters is usually denoted by Sn.

\subsection{Group Actions}

an action of a group is a formal way of interpreting the manner in which the elements of the group correspond to transformations of some space in a way that preserves the structure of that space.

. For other groups, an interpretation of the group in terms of an action may have to be specified, either because the group does not act canonically on any space or because the canonical action is not the action of interest. For example, we can specify an action of the two-element cyclic group \[ \displaystyle \mathrm {C} _{2}=\{0,1\}\] on the finite set \[\displaystyle \{a,b,c\}\] by specifying that 0 (the identity element) sends \[ \displaystyle a\mapsto a,b\mapsto b,c\mapsto c\], and that 1 sends \[ \displaystyle a\mapsto b,b\mapsto a,c\mapsto c \]. This action is not canonical.


\section{Field Theory}

In mathematics, a field is a set on which addition, subtraction, multiplication, and division are defined, and behave as the corresponding operations on rational and real numbers do. A field is thus a fundamental algebraic structure, which is widely used in algebra, number theory and many other areas of mathematics.\\
The best known fields are the field of rational numbers, the field of real numbers and the field of complex numbers. Many other fields, such as fields of rational functions, algebraic function fields, algebraic number fields, and p-adic fields are commonly used and studied in mathematics, particularly in number theory and algebraic geometry. Most cryptographic protocols rely on finite fields, i.e., fields with finitely many elements.\\
The relation of two fields is expressed by the notion of a field extension. Galois theory, initiated by Évariste Galois in the 1830s, is devoted to understanding the symmetries of field extensions. Among other results, this theory shows that angle trisection and squaring the circle can not be done with a compass and straightedge. Moreover, it shows that quintic equations are algebraically unsolvable.\\
Fields serve as foundational notions in several mathematical domains. This includes different branches of analysis, which are based on fields with additional structure. Basic theorems in analysis hinge on the structural properties of the field of real numbers. Most importantly for algebraic purposes, any field may be used as the scalars for a vector space, which is the standard general context for linear algebra. Number fields, the siblings of the field of rational numbers, are studied in depth in number theory. Function fields can help describe properties of geometric objects.\\

\section{Rings Theory}

In mathematics, a ring is one of the fundamental algebraic structures used in abstract algebra. It consists of a set equipped with two binary operations that generalize the arithmetic operations of addition and multiplication. Through this generalization, theorems from arithmetic are extended to non-numerical objects such as polynomials, series, matrices and functions.\\

Whether a ring is commutative or not (i.e., whether the order in which two elements are multiplied changes the result or not) has profound implications on its behavior as an abstract object. As a result, commutative ring theory, commonly known as commutative algebra, is a key topic in ring theory. Its development has been greatly influenced by problems and ideas occurring naturally in algebraic number theory and algebraic geometry. Examples of commutative rings include the set of integers equipped with the addition and multiplication operations, the set of polynomials equipped with their addition and multiplication, the coordinate ring of an affine algebraic variety, and the ring of integers of a number field. Examples of noncommutative rings include the ring of n × n real square matrices with n ≥ 2, group rings in representation theory, operator algebras in functional analysis, rings of differential operators in the theory of differential operators, and the cohomology ringof a topological space in topology.\\

\subsection{Polynomial Rings}

In mathematics, especially in the field of abstract algebra, a polynomial ring or polynomial algebra is a ring (which is also a commutative algebra) formed from the set of polynomials in one or more indeterminates (traditionally also called variables) with coefficients in another ring, often a field. Polynomial rings have influenced much of mathematics, from the Hilbert basis theorem, to the construction of splitting fields, and to the understanding of a linear operator. Many important conjectures involving polynomial rings, such as Serre's problem, have influenced the study of other rings, and have influenced even the definition of other rings, such as group rings and rings of formal power series.\\

A closely related notion is that of the ring of polynomial functions on a vector space. \\

\section{Iso-morphisms}

In abstract algebra, a group isomorphism is a function between two groups that sets up a one-to-one correspondence between the elements of the groups in a way that respects the given group operations. If there exists an isomorphism between two groups, then the groups are called isomorphic. From the standpoint of group theory, isomorphic groups have the same properties and need not be distinguished.


\section{Homo-morphisms}

In algebra, a homomorphism is a structure-preserving map between two algebraic structures of the same type (such as two groups, two rings, or two vector spaces). The word homomorphism comes from the ancient Greek language: ὁμός (homos) meaning "same" and μορφή (morphe) meaning "form" or "shape". However, the word was apparently introduced to mathematics due to a (mis)translation of German ähnlich meaning "similar" to ὁμός meaning "same".[1] \\

Homomorphisms of vector spaces are also called linear maps, and their study is the object of linear algebra.\\

The concept of homomorphism has been generalized, under the name of morphism, to many other structures that either do not have an underlying set, or are not algebraic. This generalization is the starting point of category theory. \\

A homomorphism may also be an isomorphism, an endomorphism, an automorphism, etc. (see below). Each of those can be defined in a way that may be generalized to any class of morphisms. \\


\section{Matrix Groups}

In mathematics, a matrix group is a group G consisting of invertible matrices over a specified field K, with the operation of matrix multiplication, and a linear group is an abstract group that is isomorphic to a matrix group over a field K, in other words, admitting a faithful, finite-dimensional representation over K.\\

Any finite group is linear, because it can be realized by permutation matrices using Cayley's theorem. Among infinite groups, linear groups form an interesting and tractable class. Examples of groups that are not linear include groups which are "too big" (for example, the group of permutations of an infinite set), or which exhibit some pathological behaviour (for example finitely generated infinite torsion groups). \\

\section{Category Theory}

Algebraic structures, with their associated homomorphisms, form mathematical categories. Category theory is a formalism that allows a unified way for expressing properties and constructions that are similar for various structures. \\

The language of category theory is used to express and study relationships between different classes of algebraic and non-algebraic objects. This is because it is sometimes possible to find strong connections between some classes of objects, sometimes of different kinds. For example, Galois theory establishes a connection between certain fields and groups: two algebraic structures of different kinds.\\