\chapter*{Abstract}

This is a report on advancements in type theory and formalized proof-based programming languages often referred to as verified programming languages. I have tried to explain the importance of mathematical modelling and logic system development based on the ancient works of mathematicians like Per-Martin Lof$^{1}$  by the applications in present programming toolchain and possible future aspects like applications in a quantum  programming toolchain.\\

Things are explained in chapter-wise manner and sufficient effort has been put to properly introduce things making understanding things easy.The chapters deal with mathematical logic systems first and then proceed to explain how this has been used and developed in real systems. With the current developments that are going on in this field, I have tried to explain the very possible usage in modelling a quantum computer's theoretical behaviour and applications in Machine Learning.\\

This area of study comes under the umbrella term of Programming Language Research and scientists all over have been  using these concepts to do explain the language semantics of any new programming language. This works using a system of inductive logics and as in abstract algebra, various operations on a given mathematical structure (\textit{example : Rings, Groups, Sets etc.}) a computer program is thought of being an operation on the a given type system. Type theory in its most literal meaning deals with the abstract idea of Types as a fundamental mathematical structures. Classical programming languages deals with data types and now some functional languages like \textit{Haskell, Agda, Ocaml etc} consider operations too as a type. Debugging has been made so easy because of the type systems and test driven development.\\

Consider the case where it is just some control signals flipping the states of bits and without a proper analytical typed contraint. The system's behaviour in this case will be completely unpredictable program-wise. Thus I would assume readers to believe with me that we would all like to have programs check that our programs are correct. Today most people who write software ,people from both academia and industry assume that the costs of formal verification of program outweighs the benefits. One simple example case is of \textit{javascript} programming language which is very informally developed thus has a very weak type system that often leads to bugs.  \textit{Haskell} on other hand does not allow programs to disobey the type system used for that program.\\ 
